%Библиография
\newpage
\addcontentsline{toc}{section}{\tocsecindent{Список литературы}}
\begin{thebibliography}{0}
\bibitem{Wiki}
Д.~Чеканов. Метод трассировки лучей против растеризации: новое поколение качества графики? ---
 Информационный портал Tom's Hardware

\bibitem{cudawiki}
Боресков А. В. Основы CUDA ---  Информационный портал steps3d.narod.ru

\bibitem{BH}
Г.Б.~Усынин, Е.В.~Кусмарцев. Реакторы на быстрых нейтронах: Учеб. пособие для вузов/Под ред. Ф.М.~Митенкова. --- 
М.: Энергоатомиздат, 1985. --- 288 с.: ил.

\bibitem{NIIAR}
Официальный сайт ОАО ``ННЦ НИИАР'' (\href{http://www.niiar.ru/}{www.niiar.ru})

\bibitem{TA}
Джон Хопкрофт, Раджив Мотвани, Джеффри Ульман Введение в теорию автоматов, языков и вычислений = Introduction to Automata Theory, Languages, and Computation. --- 
М.: Вильямс, 2002. --- 528 c. 
ISBN 0-201-44124-1

\bibitem{Nosov}
В.И.~Носов, Т.В.~Бернштейн, Н.В.~Носкова, Т.В.~Храмова. Элементы теории графов. Учебное пособие. ---
Новосибирск, 2008. --- 107 с.

\bibitem{AI}
С.Дж.~Рассел, П.~Норвиг. Искусственный интеллект: современный подход = Artificial Intelligence: A Modern Approach / Пер. с англ. и ред. К. А. Птицына. --- 2-е изд.. --- 
М.: Вильямс, 2006. --- 157—162 cc.
ISBN 5-8459-0887-6

\bibitem{habr}
Д.~Восквитцов. Реализация графов и деревьев на Python

\bibitem{PyGraph}
Гвидо ван Россум, Шаблоны Python --- реализация графов = Python Patterns - Implementing Graphs / Пер. с англ. С.Тезадов. 

\bibitem{networkX}
Официальный сайт проекта NetworkX (\href{http://networkx.lanl.gov/reference/index.html}{http://networkx.lanl.gov/reference/index.html})
\end{thebibliography}
