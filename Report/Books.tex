%Библиография
\newpage
\addcontentsline{toc}{section}{\tocsecindent{Список литературы}}
\begin{thebibliography}{0}
\bibitem{Wiki}
Д.~Чеканов. Метод трассировки лучей против растеризации: новое поколение качества графики? ---
 Информационный портал Tom's Hardware

\bibitem{cudawiki}
Боресков А. В. Основы CUDA ---  Информационный портал steps3d.narod.ru

\bibitem{BH}
Биглер Дж., Стивенс, А., и Паркер, С. Г. 2006. Разработка систем с параллельной интерактивной трассировки лучей. --- Симпозиум по интерактивной трассировки лучей, 187-196 2006.

\bibitem{Wiki}
Steven G. Parker, James Bigler, Andreas Dietrich, Heiko Friedrich, Jared Hoberock, David Luebke,  David McAllister, Morgan McGuire, Keith Morley, Austin Robison, Martin Stich. OptiX: A General Purpose Ray Tracing Engine, ACM Transactions on Graphics, 2010

\bibitem{nvidia}
NVIDIA 2010. Программирование NVIDIA OptiX движка трассировки лучей 
Руководство Версия 2.0. (\href{http://developer.nvidia.com/object/-OptiX-home.html.}{http://developer.nvidia.com/object/-OptiX-home.html.})
\end{thebibliography}
