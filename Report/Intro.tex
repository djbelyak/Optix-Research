\begin{center}
\section*{Введение}
\addcontentsline{toc}{section}{\tocsecindent{Введение}}
\end{center}

Трассировка лучей (англ. Ray tracing; рейтрейсинг) --- один из методов геометрической оптики --- исследование оптических систем путём отслеживания взаимодействия отдельных лучей с поверхностями. В узком смысле --- технология построения изображения трёхмерных моделей в компьютерных программах, при которых отслеживается обратная траектория распространения луча (от экрана к источнику).
Данный метод имеет следующие достоинства:
\begin{enumerate}
\item возможность рендеринга гладких объектов без аппроксимации их полигональными поверхностями (например, треугольниками);
\item вычислительная сложность метода слабо зависит от сложности сцены;
\item высокая алгоритмическая распараллеливаемость вычислений — можно параллельно и независимо трассировать два и более лучей, разделять участки (зоны экрана) для трассирования на разных узлах кластера и т.д;
\item отсечение невидимых поверхностей, перспектива и корректное изменения поля зрения являются логическим следствием алгоритма.
\end{enumerate}
Серьёзным недостатком метода обратного трассирования является производительность. 
Метод растеризации и сканирования строк использует когерентность данных, чтобы распределить вычисления между пикселями. 
В то время как метод трассирования лучей каждый раз начинает процесс определения цвета пикселя заново, рассматривая каждый луч наблюдения в отдельности. 
Впрочем, это разделение влечёт появление некоторых других преимуществ, таких как возможность трассировать больше лучей, чем предполагалось для устранения контурных неровностей в определённых местах модели. 
Также это регулирует отражение лучей и эффекты преломления, и в целом — степень фотореалистичности изображения.\cite{Wiki}

Для трассировки лучей NVIDIA предлагает программную библиотеку Optix, позволяющую разработчикам программного обеспечения быстро создавать приложения на основе трассировки лучей и быстро достигать результатов благодаря графическим процессорам NVIDIA и традиционным программам на языке С. 
В отличие от рендерера с неизменяемым внешним видом, ограниченного определенными структурами данных или поддерживаемым языком программирования, движок OptiX носит чрезвычайно общий характер, позволяя разработчикам программного обеспечения быстро ускорять выполнение любых задач на основе трассировки лучей и выполнять их на широко доступном оборудовании.

Целью данной учебно--исследовательской работы является создание демонстрационного приложения использованием графического движка OptiX.

Задачи, решаемые в ходе работы:
\begin{enumerate}
\item Изучение  программно-аппаратной архитектуры CUDA
\item Изучение принципов функционирования графического движка OptiX
\item Изучение процедуры установки графического движка OptiX
\item Изучение встроенных примеров графического движка OptiX
\item Разработка демонстрационного приложения
\item Выяснение перспектив применимости графического движка в прикладных приложениях
\end{enumerate}
