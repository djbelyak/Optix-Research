\documentclass[a4paper,12pt]{article}
\usepackage[T2A]{fontenc}  %поддержка кириллицы в ЛаТеХ
\addtolength{\hoffset}{-1.7mm} % горизонтальное смещение всего текста как целого
\usepackage[utf8]{inputenc}  %По умолчанию кодировка KOI8 для *nix-систем
\usepackage[english,russian]{babel} %определение языков в документе
\usepackage{amssymb,amsmath,amsfonts,latexsym,mathtext} %расширенные наборы
  % математических символов
\usepackage{cite}  %"умные" библиографические ссылки
%(сортировка и сжатие)
\usepackage{indentfirst} %делать отступ в начале параграфа
\usepackage{enumerate}  %создание и автоматическая нумерация списков
\usepackage{tabularx}  %продвинутые таблицы
%\usepackage{showkeys}  %раскомментируйте, чтобы в документе были видны
%ссылки на литературу, рисунки и таблицы
\usepackage[labelsep=period]{caption} %заменить умолчальное разделение ':' на '.'
% в подписях к рисункам и таблицам
%\usepackage[onehalfspacing]{setspace} %"умное" расстояние между строк - установить
% 1.5 интервала от нормального, эквивалентно
 \renewcommand{\baselinestretch}{1.24}
\usepackage{graphicx} %разрешить включение PostScript-графики
\graphicspath{{Images/}} %относительный путь к каталогу с рисунками,это может быть мягкая ссылка
\usepackage{listings}

\RequirePackage{caption}
\DeclareCaptionLabelSeparator{defffis}{ -- }
\captionsetup{justification=centering,labelsep=defffis}

\usepackage{geometry} %способ ручной установки полей
\geometry{top=2cm} %поле сверху
\geometry{bottom=2cm} %поле снизу
\geometry{left=2cm} %поле справа
\geometry{right=1cm} %поле слева

\RequirePackage{totcount}
\regtotcounter{section}
\regtotcounter{page}

\usepackage[colorlinks,linkcolor=blue]{hyperref}%гиперссылки в тексте
\newcommand{\tocsecindent}{\hspace{7mm}}% отступ для введения

\makeatletter
\bibliographystyle{unsrt} %Стиль библиографических ссылок БибТеХа - нумеровать
%в порядке упоминания в тексте
\renewcommand{\@biblabel}[1]{#1.}
\makeatother

\begin{document}

%Титулный лист
\begin{titlepage}
\newpage

\begin{center}
{\small\bfМИНИСТЕРСТВО ОБРАЗОВАНИЯ И НАУКИ РОССИЙСКОЙ ФЕДЕРАЦИИ\\
ОБНИНСКИЙ ИНСТИТУТ АТОМНОЙ ЭНЕРГЕТИКИ --- филиал}\\
федерального государственного автономного образовательного учреждения\\
высшего профессионального образования\\
{\bf<<Национальный исследовательский ядерный университет <<МИФИ>>\\
(ИАТЭ НИЯУ МИФИ)}\\
\vspace{2em}
Факультет кибернетики\\
Кафедра автоматизированных систем управления
\end{center}
\vspace{2em}
УДК 004.428.4
\hfill
\parbox{5.5cm}
{
ДОПУЩЕНА К ЗАЩИТЕ\\
Заведующий кафедрой АСУ\\
д.т.н., профессор\\
\hbox to 5.5cm{\dotfill А.Н. Анохин}
}
\vspace{5em}
\begin{center}
\textbf{ВЫПУСКНАЯ РАБОТА}
\end{center}

%\vspace{2em}

\begin{center}
ИССЛЕДОВАНИЕ ГРАФИЧЕСКОГО ДВИЖКА OPTIX
\end{center}

\vspace{6em}

\hbox to \textwidth
{\parbox{6 cm}{Студент гр. ИНФ-Б10}\dotfill \parbox{4 cm}{
\begin{flushright}Еличева~Е.А.\end{flushright}}}
\vspace{2em}

\hbox to \textwidth
{\parbox{6 cm}{Руководитель\\инженер НИС ИАТЭ НИЯУ МИФИ}\dotfill \parbox{4 cm}{
\begin{flushright}Белявцев~И.П.\end{flushright}}}
\vspace{2em}

\hbox to \textwidth
{\parbox{6 cm}{Рецензент\\ научный сотрудник О-ЦОД, \\
ФГБУ <<ВНИИГМИ-МЦД>>}\dotfill \parbox{4 cm}{
\begin{flushright}Кобелев~А.Е.\end{flushright}}}

\vspace{\fill}

\begin{center}
Обнинск 2014
\end{center}

\end{titlepage}


\setcounter{page}{2} % начать нумерацию с номера три
\renewcommand{\figurename}{Рисунок}

%Меняем название Оглавления
\renewcommand{\contentsname}{\centering Содержание}
%Оглавление
\tableofcontents 
\newpage


%Введение
%\begin{center}
\section*{Введение}
\addcontentsline{toc}{section}{\tocsecindent{Введение}}
\end{center}

Трассировка лучей (англ. Ray tracing; рейтрейсинг) --- один из методов геометрической оптики --- исследование оптических систем путём отслеживания взаимодействия отдельных лучей с поверхностями. В узком смысле --- технология построения изображения трёхмерных моделей в компьютерных программах, при которых отслеживается обратная траектория распространения луча (от экрана к источнику).
Данный метод имеет следующие достоинства:
\begin{enumerate}
\item возможность рендеринга гладких объектов без аппроксимации их полигональными поверхностями (например, треугольниками);
\item вычислительная сложность метода слабо зависит от сложности сцены;
\item высокая алгоритмическая распараллеливаемость вычислений — можно параллельно и независимо трассировать два и более лучей, разделять участки (зоны экрана) для трассирования на разных узлах кластера и т.д;
\item отсечение невидимых поверхностей, перспектива и корректное изменения поля зрения являются логическим следствием алгоритма.
\end{enumerate}
Серьёзным недостатком метода обратного трассирования является производительность. 
Метод растеризации и сканирования строк использует когерентность данных, чтобы распределить вычисления между пикселями. 
В то время как метод трассирования лучей каждый раз начинает процесс определения цвета пикселя заново, рассматривая каждый луч наблюдения в отдельности. 
Впрочем, это разделение влечёт появление некоторых других преимуществ, таких как возможность трассировать больше лучей, чем предполагалось для устранения контурных неровностей в определённых местах модели. 
Также это регулирует отражение лучей и эффекты преломления, и в целом — степень фотореалистичности изображения.\cite{Wiki}

Для трассировки лучей NVIDIA предлагает программную библиотеку Optix, позволяющую разработчикам программного обеспечения быстро создавать приложения на основе трассировки лучей и быстро достигать результатов благодаря графическим процессорам NVIDIA и традиционным программам на языке С. 
В отличие от рендерера с неизменяемым внешним видом, ограниченного определенными структурами данных или поддерживаемым языком программирования, движок OptiX носит чрезвычайно общий характер, позволяя разработчикам программного обеспечения быстро ускорять выполнение любых задач на основе трассировки лучей и выполнять их на широко доступном оборудовании.

Целью данной учебно--исследовательской работы является создание демонстрационного приложениея использованием графического движка OptiX.

Задачи, решаемые в ходе работы:
\begin{enumerate}
\item Изучение  программно-аппаратной архитектуры CUDA
\item Изучение принципов функционирования графического движка OptiX
\item Изучение процедуры установки графического движка OptiX
\item Изучение встроенных примеров графического движка OptiX
\item Разработка демонстрационного приложения
\item Выяснение перспектив применимости графического движка в прикладных приложениях
\end{enumerate}


% Глава 1. разработка обобщенной  модели процесса ППР реактора класса БН
%\input{Model}

% Глава 2. Алгоритмы для расчета порядка процесса ППР реактора класса БН
%\input{Graph}

%\input{Development}

%\input{Outro}

%Список литературы
%Библиография
\newpage
\addcontentsline{toc}{section}{\tocsecindent{Список литературы}}
\begin{thebibliography}{0}
\bibitem{Ukr}
С.А.~Андрушечко, А.М.~Арфов, Б.Ю.~Васильев, В.Н.~Генералов, К.Б.~Косоуров, Ю.М.~Семченков, В.Ф.~Украинцев. АЭС с реактором типа ВВЭР-1000. От физических основ эксплуатации до эволюции проекта. --- 
М.: Логос, 2010. --- 604 с.: ил.
ISBN 978-5-98704-496-4

\bibitem{BH}
Г.Б.~Усынин, Е.В.~Кусмарцев. Реакторы на быстрых нейтронах: Учеб. пособие для вузов/Под ред. Ф.М.~Митенкова. --- 
М.: Энергоатомиздат, 1985. --- 288 с.: ил.

\bibitem{NIIAR}
Официальный сайт ОАО ``ННЦ НИИАР'' (\href{http://www.niiar.ru/}{www.niiar.ru})

\bibitem{TA}
Джон Хопкрофт, Раджив Мотвани, Джеффри Ульман Введение в теорию автоматов, языков и вычислений = Introduction to Automata Theory, Languages, and Computation. --- 
М.: Вильямс, 2002. --- 528 c. 
ISBN 0-201-44124-1

\bibitem{Nosov}
В.И.~Носов, Т.В.~Бернштейн, Н.В.~Носкова, Т.В.~Храмова. Элементы теории графов. Учебное пособие. ---
Новосибирск, 2008. --- 107 с.

\bibitem{AI}
С.Дж.~Рассел, П.~Норвиг. Искусственный интеллект: современный подход = Artificial Intelligence: A Modern Approach / Пер. с англ. и ред. К. А. Птицына. --- 2-е изд.. --- 
М.: Вильямс, 2006. --- 157—162 cc.
ISBN 5-8459-0887-6

\bibitem{habr}
Д.~Восквитцов. Реализация графов и деревьев на Python

\bibitem{PyGraph}
Гвидо ван Россум, Шаблоны Python --- реализация графов = Python Patterns - Implementing Graphs / Пер. с англ. С.Тезадов. 

\bibitem{networkX}
Официальный сайт проекта NetworkX (\href{http://networkx.lanl.gov/reference/index.html}{http://networkx.lanl.gov/reference/index.html})
\end{thebibliography}




\end{document}   