\begin{center}
\section*{Реферат}
\end{center}

\vspace{2em}
39 стр., 7 рис., 6 ист.

\vspace{2em}
ГРАФИЧЕСКИЙ ДВИЖОК OPTIX, АРХИТЕКТУРА CUDA, ТРАССИРОВКА ЛУЧЕЙ, РЕНДЕРИНГ, РЕЙТРЕЙСИНГ, ОБЪЕКТНО-ОРИЕНТИРОВАННОЕ ПРОГРАММИРОВАНИЕ, {\it С++}.

\vspace{2em}
Объектом исследования является графический движок Optix.

Цель работы --- создание демонстрационного приложения с использованием графического движка Optix.

В данной работе изучается программно-аппаратная архитектура CUDA, принципы функционирования графического движка Optix.
Проводится работа по созданию демонстрационного приложения.
Приложение создается в объектно-ори\-ен\-ти\-ро\-ван\-ной парадигме программирования и на языке программирования {\it С++}.
\newpage