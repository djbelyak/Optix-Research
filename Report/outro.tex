\section{Заключение}

OptiX --- проблемно-ориентированный своевременный компилятор, который генерирует пользовательские ядра трассировки лучей, комбинируя предоставленные пользователями программы для генерации луча, заливки материала, объектного пересечения и обхода сцены. Механизм OptiX основывается на базовом наблюдении, что большинство алгоритмов трассировки лучей могут быть реализованы, используя маленький набор программируемых операций. OptiX достигает высокой производительности через компактную объектную модель и применения нескольких специфичной для трассировки лучей оптимизаций компилятора. 
Для простоты использования OptiX представляет модель программирования единственного луча с полной поддержкой рекурсии и динамического механизма отправки, подобного вызовам виртуальной функции. Механизм OptiX демонстрирует, что большинство алгоритмов трассировки лучей может быть реализовано, используя маленький набор легких программируемых операций. Механизм OptiX обеспечивает такие механизмы выполнения, что программисты пользуются знакомыми методами работы с трассировкой лучами и  не обременяют себя низкоуровневой оптимизацией. Механизм OptiX реализует объектную модель, которая использует динамическое наследование, чтобы упростить компактное представление параметров сцены. Система OptiX обеспечивает и высокоэффективный API трассировки лучей общего назначения. OptiX балансирует простоту использования с производительности, представляя простую модель программирования, на основе программируемого конвейера трассировки лучей для пользовательских программ единственного луча, которые могут быть скомпилированы в эффективное мегаядро самопланирования.Таким образом основа OptiX - JIT-компилятор, который обрабатывает программы, отрывки определенного пользователями кода на языке PTX.