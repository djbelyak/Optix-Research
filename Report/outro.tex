\section*{\centering Заключение}
\addcontentsline{toc}{section}{\tocsecindent{Заключение}}

В ходе данной работы была изучена  программно-аппаратная архитектура CUDA,было выяснено, что архитектура CUDA является довольно простой, но в тоже время мощной технологией для массивно-параллельных вычислений. С помощью архитектуры CUDA практически в режиме реального времени решать задачи по поиску пути в графе, аппроксимировать функции, обрабатывать большие массивы данных и т.д.

 Так же подробно были рассмотренны принципы функционирования графического движка OptiX. Были рассмотрены примеры, входящие в набор разработчика OptiX. На основе полученной информации было разработано демонстрационное приложение. 

Таким образом были получены достаточная информация и практическое подтверждение того, что Optix является простым, но в тоже время быстрым движком для трассировки лучей общего назначения. Optix можно использовать не только для фотореалистичного рендеринга 3D изображения, но и для задач моделирования корпускулярных и волновых процессов. Это значит, что Optix можно использовать в процессах моделирования звукового распространения, оптических эффектов и так же моделирование ядерных процессов. 