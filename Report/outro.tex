\section{Заключение}
Проект NVIDIA OptiX имеет цель--- создание движка реального времени, осуществляющего рендеринг методом трассировки лучей, опираясь на вычислительную мощность графических процессоров NVIDIA. Первое поколение движка уже обеспечивало как вполне себе рендеринг в реальном времени для относительно простых сцен.
Трассировка лучей по сравнению с растровыми методами (таковым, например, является метод использования OpenGL) предоставляет результаты на порядок более высокого качества. Отсутствие дефектов, идеальная прорисовка всех оптических свойств должны были сделать этот алгоритм наиболее популярным. На самом же деле трассировка применяется весьма редко и только там, где качество визуализации — несравненно более важный критерий, нежели время выполнения. Поэтому вопрос об оптимизации алгоритма стоит очень остро.

Вначале растровыми средствами выстраивается весьма грубая картинка, а затем на нее сверху добавляются отдельные специальные эффекты методом трассировки. 
В анимации используется информация о предыдущем кадре. Вначале картинка строится только по сохраненной информации с учетом изменения угла обзора, расстояния до объектов и т.д. В новом кадре при этом возникают «черные дыры», или коллизии, обнаружение которых возлагается на небольшой эвристический анализ, который, к примеру, сравнивает яркость соседних пикселов, выставляя каждой точке изображения уровень «ошибочности», а затем для наиболее «ошибочных» пикселов вызывается новая трассировка.
Уменьшить количество обрабатываемых лучей можно, интерполируя цвет пикселов по ближайшему окружению. Чтобы избежать дефектов, нужно прогнозировать расположение областей с высокой контрастностью и просчитывать их точно, без аппроксимации.