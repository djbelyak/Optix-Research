\section*{\centering Заключение}
\addcontentsline{toc}{section}{\tocsecindent{Заключение}}

В ходе данной работы изучено следующее:
\begin{itemize}
\item Программно-аппаратная архитектура CUDA. Архитектура CUDA является довольно простой, но в тоже время мощной технологией для массивно-параллельных вычислений. С помощью архитектуры CUDA практически в режиме реального времени решать задачи по поиску пути в графе, аппроксимировать функции, обрабатывать большие массивы данных и т.д.

 \item Рассмотренны принципы функционирования графического движка OptiX. Он включает реализацию очень широкого набора основанных на трассировке лучей алгоритмов и приложений, включая интерактивный рендеринг, оффлайн рендеринг, системы обнаружения коллизий, запросы искусственного интеллекта и научного моделирования, такие как звуковое распространение.
 Механизм OptiX фокусируется исключительно на фундаментальных вычислениях, требуемых для трассировки лучей.
 Механизм OptiX демонстрирует, что большинство алгоритмов трассировки лучей могут быть реализованы, используя маленький набор легких программируемых операций. Механизм OptiX комбинирует своевременные методы компиляции со специфичным для трассировки лучей знанием, чтобы реализовать его модель программирования эффективно. 
 \item Рассмотрены примеры, входящие в набор разработчика OptiX.
 
 \item На основе полученной информации разработано демонстрационное приложение. 
\end{itemize}
Таким образом  получены достаточная информация и практическое подтверждение того, что Optix является простым, но в тоже время быстрым движком для трассировки лучей общего назначения. Optix можно использовать не только для фотореалистичного рендеринга 3D изображения, но и для задач моделирования корпускулярных и волновых процессов. Это значит, что Optix можно использовать в процессах моделирования звукового распространения, оптических эффектов и так же моделирования ядерных процессов. 