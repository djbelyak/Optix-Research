\section{Создание демонстрационного приложения}
\subsection{Исследование примеров}
 Рассмотрим пример glass из пакета разработчика Optix. 
 Пример представляет собой класс GlassScene с рядом сопроводительных CUDA процедур и обработчиков запуска.
 Обработчиком запуска является процедура main:
 \begin{verbatim}
 int main(int argc, char* argv[])
{
  GLUTDisplay::init( argc, argv );

  bool adaptive_aa = true;  // Default to true for now
  bool green_glass = false;
  std::string obj_path;
  for ( int i = 1; i < argc; ++i ) {
    std::string arg( argv[i] );
    if ( arg == "--adaptive-off" || arg == "-A" ) {
      adaptive_aa = false;
    } else if ( arg == "--green" || arg == "-g" ) {
      green_glass = true;
    } else if ( arg == "--obj-path" || arg == "-o" ) {
      if ( i == argc-1 ) {
        printUsageAndExit( argv[0] );
      }
      obj_path = argv[++i];
    } else if ( arg == "--help" || arg == "-h" ) {
      printUsageAndExit( argv[0] );
    } else {
      std::cerr << "Unknown option: '" << arg << "'\n";
      printUsageAndExit( argv[0] );
    }
  }

  if( !GLUTDisplay::isBenchmark() ) printUsageAndExit( argv[0], false );

  if( obj_path.empty() ) {
    obj_path = std::string( sutilSamplesDir() ) + "/glass";
  }

  try {
    GlassScene scene( obj_path, adaptive_aa, green_glass );
    GLUTDisplay::setTextColor( make_float3( 0.2f ) );
    GLUTDisplay::setTextShadowColor( make_float3( 0.9f ) );
    GLUTDisplay::run( "GlassScene", &scene, adaptive_aa ? GLUTDisplay::CDProgressive : GLUTDisplay::CDNone );
  } catch( Exception& e ){
    sutilReportError( e.getErrorString().c_str() );
    exit(1);
  }

  return 0;
}
\end{verbatim}
Как  видно из исходного кода,  обработчик запуска создает объект класса GlassScene инициализирует его параметрами командной строки и передает его процессору запуска GLUTDisplay. Для корректной работы с GLUTDisplay класс GlassScene должен иметь следующие методы:
\begin{verbatim}
void   initScene( InitialCameraData& camera_data );
  void   trace( const RayGenCameraData& camera_data );
  void   doResize( unsigned int width, unsigned int depth );
  Buffer getOutputBuffer();
  bool keyPressed(unsigned char key, int x, int y);
\end{verbatim}
 InitScene создает объект сцены внутри контекста трассировки. 
 trace используется для построения изображения методом трассировки лучей.
 doResize производит необходимые изменения контекста для изменения получаемого изображения.
 getOutputBuffer создается выходной буфер, содержащий выходное изображение.
  keyPressed--- обработчик клавиш.
  Результат работы данного примера можно увидеть на рисунке\ref{glass}
\begin{figure}[h!]
\center{\includegraphics[width=0.5\linewidth]{glass.png}}
\caption{пример 1.}
\label{glass}
\end{figure}
 
\subsection{Выполнение работы}
\begin{verbatim}
#include <optixu/optixpp_namespace.h>
#include <optixu/optixu_math_namespace.h>
#include <optixu/optixu_aabb_namespace.h>
#include <sutil.h>
#include <GLUTDisplay.h>
#include <PlyLoader.h>
#include <ObjLoader.h>
#include "commonStructs.h"
#include <string>
#include <iostream>
#include <fstream>
#include <cstdlib>
#include <cstring>
#include "random.h"
#include "MeshScene.h"

using namespace optix;

class MeshViewer : public MeshScene
{
public:
  enum ShadeMode
  {
    SM_PHONG=0,
    SM_AO,
    SM_NORMAL,
    SM_ONE_BOUNCE_DIFFUSE,
    SM_AO_PHONG
  };
  enum CameraMode
  {
    CM_PINHOLE=0,
    CM_ORTHO
  };
    MeshViewer();
  // Setters for controlling application behavior
  void setShadeMode( ShadeMode mode ) { m_shade_mode = mode;}
  void setCameraMode( CameraMode mode ) { m_camera_mode = mode;}
  void setAORadius( float ao_radius ) { m_ao_radius = ao_radius;}
  void setAOSampleMultiplier(
   int ao_sample_mult ) { m_ao_sample_mult = ao_sample_mult; }
  void setLightScale( float light_scale ) { m_light_scale = light_scale;}
  void setAA( bool onoff ) { m_aa_enabled = onoff;}
  void setAnimation( bool anim ) { m_animation = anim;}

  virtual void   initScene( InitialCameraData& camera_data );
  virtual void   doResize( unsigned int width, unsigned int height );
  virtual void   trace( const RayGenCameraData& camera_data );
  virtual void   cleanUp();
  virtual bool   keyPressed(unsigned char key, int x, int y);
  virtual Buffer getOutputBuffer();

private:
  void initContext();
  void initLights();
  void initMaterial();
  void initGeometry();
  void initCamera( InitialCameraData& cam_data );
  void preprocess();
  void resetAccumulation();
  void genRndSeeds( unsigned int width, unsigned int height );

  CameraMode    m_camera_mode;
  ShadeMode  m_shade_mode;
  bool    m_aa_enabled;
  float    m_ao_radius;
  int       m_ao_sample_mult;
  float    m_light_scale;

  Material  m_material;
  Aabb      m_aabb;
  Buffer     m_rnd_seeds;
  Buffer     m_accum_buffer;
  bool       m_accum_enabled;

  float       m_scene_epsilon;
  int          m_frame;
  bool        m_animation;
};
MeshViewer::MeshViewer():
  MeshScene ( false, false, false ),
  m_camera_mode ( CM_PINHOLE ),
  m_shade_mode ( SM_PHONG ),
  m_aa_enabled ( false ),
  m_ao_radius ( 1.0f ),
  m_ao_sample_mult ( 1 ),
  m_light_scale ( 1.0f ),
  m_accum_enabled ( false ),
  m_scene_epsilon ( 1e-4f ),
  m_frame ( 0 ),
  m_animation ( false )
{
}
void MeshViewer::initScene( InitialCameraData& camera_data )
{
  initContext();
  initLights();
  initMaterial();
  initGeometry();
  initCamera( camera_data );
  preprocess();
}
void MeshViewer::initContext()
{
  m_context->setRayTypeCount( 3 );
  m_context->setEntryPointCount( 1 );
  m_context->setStackSize( 1180 );

  m_context[ "radiance_ray_type" ]->setUint( 0u );
  m_context[ "shadow_ray_type" ]->setUint( 1u );
  m_context[ "max_depth" ]->setInt( 5 );
  m_context[ "ambient_light_color" ]->setFloat( 0.2f, 0.2f, 0.2f );
  m_context[ "output_buffer" ]->set(
   createOutputBuffer(RT_FORMAT_UNSIGNED_BYTE4, WIDTH, HEIGHT) );
  m_context[ "jitter_factor" ]->setFloat( m_aa_enabled ? 1.0f : 0.0f );
  
  m_accum_enabled = m_aa_enabled ||
                   m_shade_mode == SM_AO ||
                   m_shade_mode == SM_ONE_BOUNCE_DIFFUSE ||
                   m_shade_mode == SM_AO_PHONG;
 // Ray generation program setup
  const std::string camera_name = m_camera_mode == CM_PINHOLE ?
   "pinhole_camera" : "orthographic_camera"; 
  const std::string camera_file = m_accum_enabled? "accum_camera.cu" :
                         m_camera_mode == CM_PINHOLE ?
                       "pinhole_camera.cu"  : "orthographic_camera.cu";

  if( m_accum_enabled ) {
    // The raygen program needs accum_buffer
    m_accum_buffer = m_context->createBuffer(
     RT_BUFFER_INPUT_OUTPUT | RT_BUFFER_GPU_LOCAL, RT_FORMAT_FLOAT4,
                                            WIDTH, HEIGHT );
    m_context["accum_buffer"]->set( m_accum_buffer );
    resetAccumulation();
  }
  const std::string camera_ptx  = ptxpath( "sample6", camera_file );
  Program ray_gen_program = m_context->createProgramFromPTXFile( 
  camera_ptx, camera_name );
  m_context->setRayGenerationProgram( 0, ray_gen_program );
  // Exception program
  const std::string except_ptx  = ptxpath( "sample6", camera_file );
  m_context->setExceptionProgram( 0, m_context->createProgramFromPTXFile( 
  except_ptx, "exception" ) );
  m_context[ "bad_color" ]->setFloat( 0.0f, 1.0f, 0.0f );
 // Miss program 
  const std::string miss_ptx = ptxpath( "sample6", "constantbg.cu" );
  m_context->setMissProgram( 0, m_context->createProgramFromPTXFile( 
  miss_ptx, "miss" ) );
  m_context[ "bg_color" ]->setFloat(  0.34f, 0.55f, 0.85f );
}
void MeshViewer::initLights()
{
  // Lights buffer
  BasicLight lights[] = {
    { make_float3( -60.0f,  30.0f, -120.0f ),
     make_float3( 0.2f, 0.2f, 0.25f )*m_light_scale, 0, 0 },
    { make_float3( -60.0f,   0.0f,  120.0f ),
     make_float3( 0.1f, 0.1f, 0.10f )*m_light_scale, 0, 0 },
    { make_float3(  60.0f,  60.0f,   60.0f ), 
    make_float3( 0.7f, 0.7f, 0.65f )*m_light_scale, 1, 0 }
  };
 Buffer light_buffer = m_context->createBuffer(RT_BUFFER_INPUT);
  light_buffer->setFormat(RT_FORMAT_USER);
  light_buffer->setElementSize(sizeof( BasicLight ) );
  light_buffer->setSize( sizeof(lights)/sizeof(lights[0]) );
  memcpy(light_buffer->map(), lights, sizeof(lights));
  light_buffer->unmap();

  m_context[ "lights" ]->set( light_buffer );
}
void MeshViewer::initMaterial()
{
  switch( m_shade_mode ) {
    case SM_PHONG: {
      // Use the default obj_material created by ObjLoader
      break;
    }
   case SM_NORMAL: {
      const std::string ptx_path = ptxpath("sample6", "normal_shader.cu");
      m_material = m_context->createMaterial();
      m_material->setClosestHitProgram(
       0, m_context->createProgramFromPTXFile( ptx_path, "closest_hit_radiance" ) );
      break;
    }
    case SM_AO: {
      const std::string ptx_path = ptxpath("sample6", "ambocc.cu");
      m_material = m_context->createMaterial();
      m_material->setClosestHitProgram(
       0, m_context->createProgramFromPTXFile( ptx_path, "closest_hit_radiance" ) );
      m_material->setAnyHitProgram(
       1, m_context->createProgramFromPTXFile( ptx_path, "any_hit_occlusion" ) );    
      break;
    }  
    case SM_ONE_BOUNCE_DIFFUSE: {
      const std::string ptx_path = ptxpath("sample6", "one_bounce_diffuse.cu");
      m_material = m_context->createMaterial();
      m_material->setClosestHitProgram(
       0, m_context->createProgramFromPTXFile( ptx_path, "closest_hit_radiance" ) );
      m_material->setAnyHitProgram    (
       1, m_context->createProgramFromPTXFile( ptx_path, "any_hit_shadow" ) );
      break;
    }
    case SM_AO_PHONG: {
      const std::string ptx_path = ptxpath("sample6", "ambocc.cu");
      m_material = m_context->createMaterial();
      m_material->setClosestHitProgram( 
      0, m_context->createProgramFromPTXFile( 
      ptx_path, "closest_hit_radiance_phong_ao" ) );
      m_material->setAnyHitProgram(
       1, m_context->createProgramFromPTXFile( ptx_path, "any_hit_shadow" ) );
      m_material->setAnyHitProgram(
       2, m_context->createProgramFromPTXFile( ptx_path, "any_hit_occlusion" ) );
      m_context["Kd"]->setFloat(1.0f);
      m_context["Ka"]->setFloat(0.6f);
      m_context["Ks"]->setFloat(0.0f);
      m_context["Kr"]->setFloat(0.0f);
      m_context["phong_exp"]->setFloat(0.0f);
      break;
    }
  }
  if( m_accum_enabled ) {
    genRndSeeds( WIDTH, HEIGHT );
  }
}
void MeshViewer::initGeometry()
{
  double start, end;
  sutilCurrentTime(&start);

  m_geometry_group = m_context->createGeometryGroup();
  if( ObjLoader::isMyFile( m_filename.c_str() ) ) {
    // Load OBJ model 
    ObjLoader* loader = 0;
    if( m_shade_mode == SM_NORMAL || m_shade_mode == SM_AO |
    | m_shade_mode == SM_AO_PHONG ) {
      loader = new ObjLoader( m_filename.c_str(), m_context, m_geometry_group,
       m_material, false,  m_accel_builder.c_str(), m_accel_traverser.c_str(),
        m_accel_refine.c_str(), m_accel_large_mesh );
    } else if ( m_shade_mode == SM_ONE_BOUNCE_DIFFUSE ) {
      loader = new ObjLoader( m_filename.c_str(), m_context, m_geometry_group,
       m_material, true,  m_accel_builder.c_str(), m_accel_traverser.c_str(), 
       m_accel_refine.c_str(), m_accel_large_mesh );
    } else {
      loader = new ObjLoader( m_filename.c_str(), m_context, m_geometry_group,
       m_accel_builder.c_str(), m_accel_traverser.c_str(),
        m_accel_refine.c_str(), m_accel_large_mesh );
    }
    loader->load();
    m_aabb = loader->getSceneBBox();
    delete loader;
  } else if( PlyLoader::isMyFile( m_filename ) ) {
    // Load PLY model 
    PlyLoader loader( m_filename, m_context, m_geometry_group, m_material,
     m_accel_builder.c_str(), m_accel_traverser.c_str(), m_accel_refine.c_str(), 
     m_accel_large_mesh );
    loader.load();
    m_aabb = loader.getSceneBBox();
  } else {
    std::cerr << "Unrecognized model file extension '" <<
     m_filename << "'" << std::endl;
    exit( 0 );
  }
  loadAccelCache();
 
  m_context[ "top_object" ]->set( m_geometry_group );
  m_context[ "top_shadower" ]->set( m_geometry_group );

  sutilCurrentTime(&end);
  std::cerr << "Time to load " << (m_accel_large_mesh ? "and cluster " : "") <<
   "geometry: " << end-start << " s.\n";
}
void MeshViewer::initCamera( InitialCameraData& camera_data )
{
  // Set up camera
  float max_dim  = m_aabb.maxExtent();
  float3 eye     = m_aabb.center();
  eye.z         += 2.0f * max_dim;

  camera_data = InitialCameraData( eye,  // eye
                                   m_aabb.center(), // lookat
                                   make_float3( 0.0f, 1.0f, 0.0f ), // up
                                   30.0f );                         // vfov

  // Declare camera variables. 
  m_context[ "eye"]->setFloat( make_float3( 0.0f, 0.0f, 0.0f ) );
  m_context[ "U"  ]->setFloat( make_float3( 0.0f, 0.0f, 0.0f ) );
  m_context[ "V"  ]->setFloat( make_float3( 0.0f, 0.0f, 0.0f ) );
  m_context[ "W"  ]->setFloat( make_float3( 0.0f, 0.0f, 0.0f ) );
}
void MeshViewer::preprocess()
{
  // Settings which rely on previous initialization
  m_scene_epsilon = 1.e-4f * m_aabb.maxExtent();
  m_context[ "scene_epsilon"      ]->setFloat( m_scene_epsilon );
  m_context[ "occlusion_distance" ]->setFloat(
   m_aabb.maxExtent() * 0.3f * m_ao_radius );

  // Prepare to run 
  m_context->validate();
  double start, end_compile, end_AS_build;
  sutilCurrentTime(&start);
  m_context->compile();
  sutilCurrentTime(&end_compile);
  std::cerr << "Time to compile kernel: "<<end_compile-start<<" s.\n";
  m_context->launch(0,0);
  sutilCurrentTime(&end_AS_build);
  std::cerr << "Time to build AS: "<<end_AS_build-end_compile<<" s.\n";
  // Save cache file
  saveAccelCache();
}
bool MeshViewer::keyPressed(unsigned char key, int x, int y)
{
   switch (key)
   {
     case 'e':
       m_scene_epsilon *= .1f;
       std::cerr << "scene_epsilon: " << m_scene_epsilon << std::endl;
       m_context[ "scene_epsilon" ]->setFloat( m_scene_epsilon );
       return true;
     case 'E':
       m_scene_epsilon *= 10.0f;
       std::cerr << "scene_epsilon: " << m_scene_epsilon << std::endl;
       m_context[ "scene_epsilon" ]->setFloat( m_scene_epsilon );
       return true;
   }
   return false;
}
void MeshViewer::doResize( unsigned int width, unsigned int height )
{
  // output_buffer resizing handled in base class
  if( m_accum_enabled ) {
    m_accum_buffer->setSize( width, height );
    m_rnd_seeds->setSize( width, height );
    genRndSeeds( width, height );
    resetAccumulation();
  }
}
void MeshViewer::trace( const RayGenCameraData& camera_data )
{
  if (m_animation && GLUTDisplay::isBenchmark() ) {
    static float angleU = 0.0f, angleV = 0.0f, scale = 1.0f,
     dscale = 0.96f, backside = 0.0f;
    static int phase = 0, accumed_frames = 0;
    const float maxang = M_PIf * 0.2f;
    const float rotvel = M_2_PIf*0.1f;
    float3 c = m_aabb.center();
    float3 e = camera_data.eye;

    Matrix3x3 m = make_matrix3x3(
    Matrix4x4::rotate(angleV + backside, normalize(camera_data.V)) * 
      Matrix4x4::rotate(angleU, normalize(camera_data.U)) * Matrix4x4::scale(
      make_float3(scale, scale, scale)));

    if( !m_accum_enabled || accumed_frames++ > 5 ) {
      switch(phase) {
      case 0: angleV += rotvel; if(angleV > maxang)
       { angleV =  maxang; phase++; } break;
      case 1: angleU += rotvel; if(angleU > maxang)
       { angleU =  maxang; phase++; } break;
      case 2: angleV -= rotvel; if(angleV <-maxang)
       { angleV = -maxang; phase++; } break;
      case 3: angleU -= rotvel; if(angleU <-maxang)
       { angleU = -maxang; phase=0; } break;
      }
      scale *= dscale;
      if(scale < 0.1f) { dscale = 1.0f / dscale; backside = M_PIf - backside; }
      if(scale > 1.0f) { dscale = 1.0f / dscale; }

      accumed_frames = 0;
      m_camera_changed = true;
    }

    m_context["eye"]->setFloat( c-m*(c-e) );
    m_context["U"]->setFloat( m*camera_data.U );
    m_context["V"]->setFloat( m*camera_data.V );
    m_context["W"]->setFloat( m*camera_data.W );
  } else {
    m_context["eye"]->setFloat( camera_data.eye );
    m_context["U"]->setFloat( camera_data.U );
    m_context["V"]->setFloat( camera_data.V );
    m_context["W"]->setFloat( camera_data.W );
  }
  Buffer buffer = m_context["output_buffer"]->getBuffer();
  RTsize buffer_width, buffer_height;
  buffer->getSize( buffer_width, buffer_height );

  if( m_accum_enabled && !m_camera_changed ) {
    m_context["sqrt_occlusion_samples"]->setInt( 3 * m_ao_sample_mult );
    m_context["sqrt_diffuse_samples"]->setInt( 3 );
  }
  m_context->launch( 0, static_cast<unsigned int>(buffer_width),
   static_cast<unsigned int>(buffer_height) );

  if( m_accum_enabled ) {
      ++m_frame;
    if( m_camera_changed ) {
      m_camera_changed = false;
      resetAccumulation();
    }
    m_context["frame"]->setInt( m_frame );
  }
}
void MeshViewer::cleanUp()
{
  SampleScene::cleanUp();
}
Buffer MeshViewer::getOutputBuffer()
{
  return m_context["output_buffer"]->getBuffer();
}
void MeshViewer::resetAccumulation()
{
  m_frame = 0;
  m_context[ "frame"]->setInt( m_frame );
  m_context[ "sqrt_occlusion_samples" ]->setInt( 1 * m_ao_sample_mult );
  m_context[ "sqrt_diffuse_samples"   ]->setInt( 1 );
}
void MeshViewer::genRndSeeds( unsigned int width, unsigned int height )
{
  // Init random number buffer if necessary.
  if( m_rnd_seeds.get() == 0 ) {
    m_rnd_seeds = m_context->createBuffer(
     RT_BUFFER_INPUT_OUTPUT | RT_BUFFER_GPU_LOCAL, 
     RT_FORMAT_UNSIGNED_INT, WIDTH, HEIGHT);
    m_context["rnd_seeds"]->setBuffer(m_rnd_seeds);
  }
  unsigned int* seeds = static_cast<unsigned int*>( m_rnd_seeds->map() );
  fillRandBuffer(seeds, width*height);
  m_rnd_seeds->unmap();
}
void printUsageAndExit( const std::string& argv0, bool doExit = true )
{
  std::cerr
    << "Usage  : " << argv0 << " [options]\n"
    << "App options:\n"
    << "  -h  | --help                               
    << "  -o  | --obj <obj_file>                    
    << "  -c  | --cache                              
    << "  -a  | --ao-shade                              
    << "  -ap | --ao-phong-shade                    
    << "  -aa | --antialias                          
    << "  -n  | --normal-shade                   
    << "  -i  | --diffuse-shade                    
    << "  -O  | --ortho                              
    << "  -r  | --ao-radius <scale>              
    << "  -m  | --ao-sample-mult <n>             
    << "  -l  | --light-scale <scale>          
    << "        --large-mesh                 
    << "        --animation                    
    << "        --trav <name>          
    << "        --build <name>       
    << "        --refine <n>         
    << std::endl;
  GLUTDisplay::printUsage();

  std::cerr
    << "App keystrokes:\n"
    << "  e Decrease scene epsilon size (used for shadow ray offset)\n"
    << "  E Increase scene epsilon size (used for shadow ray offset)\n"
    << std::endl;

  if ( doExit ) exit(1);
}
int main( int argc, char** argv ) 
{
  GLUTDisplay::init( argc, argv );
  
  GLUTDisplay::contDraw_E draw_mode = GLUTDisplay::CDNone; 
  MeshViewer scene;
  scene.setMesh( (std::string( sutilSamplesDir() ) +
   "/simpleAnimation/cow.obj").c_str() );

  for ( int i = 1; i < argc; ++i ) {
    std::string arg( argv[i] );
    if ( arg == "-c" || arg == "--cache" ) {
      scene.setAccelCaching( true );
    } else if( arg == "-n" || arg == "--normal-shade" ) {
      scene.setShadeMode( MeshViewer::SM_NORMAL );
    } else if( arg == "-a" || arg == "--ao-shade" ) {
      scene.setShadeMode( MeshViewer::SM_AO);
      draw_mode = GLUTDisplay::CDProgressive;
    } else if( arg == "-i" || arg == "--diffuse-shade" ) {
      scene.setShadeMode( MeshViewer::SM_ONE_BOUNCE_DIFFUSE );
      draw_mode = GLUTDisplay::CDProgressive;
    } else if( arg == "-ap" || arg == "--ao-phong-shade" ) {
      scene.setShadeMode( MeshViewer::SM_AO_PHONG );
      draw_mode = GLUTDisplay::CDProgressive;
    } else if( arg == "-aa" || arg == "--antialias" ) {
      scene.setAA( true );
      draw_mode = GLUTDisplay::CDProgressive;
    } else if( arg == "-O" || arg == "--ortho" ) {
      scene.setCameraMode( MeshViewer::CM_ORTHO );
    } else if( arg == "-h" || arg == "--help" ) {
      printUsageAndExit( argv[0] ); 
    } else if( arg == "-o" || arg == "--obj" ) {
      if ( i == argc-1 ) {
        printUsageAndExit( argv[0] );
      }
      scene.setMesh( argv[++i] );
    } else if( arg == "--trav" ) {
      if ( i == argc-1 ) {
        printUsageAndExit( argv[0] );
      }
      scene.setTraverser( argv[++i] );
    } else if( arg == "--build" ) {
      if ( i == argc-1 ) {
        printUsageAndExit( argv[0] );
      }
      scene.setBuilder( argv[++i] );
    } else if( arg == "--refine" ) {
      if ( i == argc-1 ) {
        printUsageAndExit( argv[0] );
      }
      scene.setRefine( argv[++i] );
    } else if( arg == "--kd" ) {  
      scene.setBuilder( "TriangleKdTree" );
      scene.setTraverser( "KdTree" );
    } else if( arg == "--lbvh" ) { 
      scene.setBuilder( "Lbvh" );
    } else if( arg == "--bvh" ) {
      scene.setBuilder( "Bvh" );
    } else if( arg == "--large-mesh" ) {
      scene.setLargeMesh( true );
    } else if( arg == "--animation" ) {
      scene.setAnimation( true );
    } else if( arg == "-r" || arg == "--ao-radius" ) {
      if ( i == argc-1 ) {
        printUsageAndExit( argv[0] );
      }
      scene.setAORadius( static_cast<float>( atof( argv[++i] ) ) );
    } else if( arg == "-m" || arg == "--ao-sample-mult" ) {
      if ( i == argc-1 ) {
        printUsageAndExit( argv[0] );
      }
      scene.setAOSampleMultiplier( atoi( argv[++i] ) );
    } else if( arg == "-l" || arg == "--light-scale" ) {
      if ( i == argc-1 ) {
        printUsageAndExit( argv[0] );
      }
      scene.setLightScale( static_cast<float>( atof( argv[++i] ) ) );
    } else {
      std::cerr << "Unknown option: '" << arg << "'" << std::endl;
      printUsageAndExit( argv[0] );
    }
  }  
  if( !GLUTDisplay::isBenchmark() ) printUsageAndExit( argv[0], false );

  try {
    GLUTDisplay::run( "OptiX Viewer", &scene, draw_mode );
  } catch( Exception& e ){
    sutilReportError( e.getErrorString().c_str() );
    exit(1);
  }
  return 0;
}
\end{verbatim}

Результат работы:
\begin{figure}[h!]
\center{\includegraphics[width=0.5\linewidth]{my.png}}
\caption{Пример работы написанного приложения}
\label{map}
\end{figure}